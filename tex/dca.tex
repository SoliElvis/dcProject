\newcommand{\xs}{x^\nu}
\newcommand{\xss}{x^{\nu+1}}
\newcommand{\ys}{y^{\nu}}
\newcommand{\yss}{y^{\nu+1}}
\subsection{Intuition and Derivation of simplified DCA}
We now have the technical results necessary to discuss the
DC algorithm for general D.C. programs. Its aim is to 
generate sequences $x^\nu$ and $y^\nu$ which converge to a local
minimum of the functions $f:=g-h$ and $f^\dagger:= h^*-g^*$
respectively where $g,h\in\Gamma_0$\\ We define two sub programs:
\begin{align} 
 \widetilde S(x) \qquad &\inf\{h^*(y)-g^*(y) \ : \ y\in\partial h(x)\}\\ 
 \widetilde T(y) \qquad &\inf\{g(x)-h(x) \ : \ x\in\partial g^*(y)\}
 \label{fulldca}
\end{align}
Let $\widetilde {\mathcal S}(x)$ and $\widetilde{\mathcal T}(y)$ denote the
solution sets of $\widetilde S(x)$ and $\widetilde T(y)$ respectively. Then given
a point $x^o\in\ddom g$
the sequences are constructed as follows:
\begin{equation}
  \ys\in\widetilde{\mathcal S}(\xs); \quad \xss\in\widetilde{\mathcal T}(\ys)
\end{equation}
The main idea relies on a decomposition approach to the problem;
we use DC duality to solve the initial program and its dual
on subsets of the initial space using the subdifferential
operators to move between primal and dual. Eventhough the 
the subprograms are simpler than the initial one they are 
still non-convex. Their main function is to give 
insights as to how to construct sequences with the desired
behaviour. \\

Consider the program $T(\ys)$ where $\ys$ is the solution
to some $\widetilde S(x^o)$. We have that $\ys\in\partial h(\xs)$ hence:
\begin{equation}
    h(\xs)\leq h(\xss) + \inp{y}{x-\xss} 
\end{equation}
This suggests a simplfication of the program $\widetilde T(\ys)$ by replacing
$h(\xs)$ by its affine minorization through $\ys$ whereby we get the
following program, which is convex: 
\begin{equation}
    \inf\{g(x) - [h(\xs)+\inp{\ys}{x-\xs}] \}
    \label{simDCApp}
\end{equation}
Dually we can also replace $g^*$ in the objective function
of $\widetilde S(\xss)$ by its affine minorant since we have that
$\xss \in \partial g^*(\ys)$:
\begin{equation}
    \inf\{h^*(y)-[g^*(\ys) + \inp{\xss}{y-\ys}]\}
    \label{simDCAdd}
\end{equation}
Note that we only needed the inclusion $\xss\in\partial g^*(\ys)$
and $\ys\in\partial h(\xs)$ to find a simplified versions of
the two initial programs.\\
\begin{comment}
Note that the simplified forms \eqref{simDCApp} and \eqref{simDCAdd}
are actually equivalent to the programs from \eqref{fulldca} if the functions 
$g^*$ and $h$ are actually essentially differentiable \autocite{tao2005dc}.
\end{comment}
\newpage
%
The programs \eqref{simDCApp} and \eqref{simDCAdd} can actually be simplified
further as the objective functions contain constants.
%
\begin{equation}
   \argmin\{h^*(y)-[g^*(\ys) + \inp{\xss}{y-\ys}]\}=
   \argmin\{h^*(y) -\inp{\xss}{y}\}
\end{equation}
Now note that 
\begin{equation}
  \yss \in \argmin\{h^*(y) -\inp{\xss}{y}\} \iff \yss \in \partial h(\xss)
\end{equation}
Similarly
\begin{equation}
    \argmin\{g(x) - [h(\xs)+\inp{\ys}{x-\xs}] \}=
   \argmin\{g(x) -\inp{x}{\ys}\}
\end{equation}
\begin{equation}
  \xss \in   \argmin\{g(x) -\inp{x}{\ys}\} \iff \xss \in \partial g^*\ys)
\end{equation}
Hence all in all the simplified DCA 
