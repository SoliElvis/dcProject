\documentclass{article}

\usepackage{comment}
\usepackage[english]{isodate}
\usepackage{graphicx}
\usepackage[margin=1in]{geometry}
\usepackage{paracol}
\usepackage[utf8]{inputenc}
\usepackage[T1]{fontenc}
\usepackage[bookmarks=true]{hyperref}
\usepackage{bookmark}
\usepackage{pdfpages}
%\includepdf[pages={1}]{myfile.pdf}

%quotes
\usepackage{csquotes} 
\usepackage{endnotes}
\usepackage[toc,page]{appendix}


\usepackage[backend=biber,style=authoryear]{biblatex} 
\usepackage[toc,page]{appendix}
\usepackage{bookmark} 

\renewcommand*{\thefootnote}{\alph{footnote}}

\usepackage[belowskip=-15pt,aboveskip=0pt]{caption,subcaption}
\usepackage{cancel}
\usepackage{listings}
\usepackage{fancyhdr}
\usepackage{lastpage}
\pagestyle{fancy}
\fancyfoot{}
\fancyfoot[R]{\thepage\ of \pageref{LastPage}}

\usepackage{mathtools,xparse}
\usepackage{amsmath}
\usepackage{amssymb}
\usepackage{siunitx}
\usepackage{amsthm}
\usepackage{amsbsy}

\usepackage{bm}
\newcommand{\ddom}{\mathrm{dom} \, }
\DeclarePairedDelimiterX{\inp}[2]{\langle}{\rangle}{#1, #2}

\usepackage{mdframed}
\mdfdefinestyle{exampledefault}{%
rightline=true,leftline=true, innerleftmargin=3,innerrightmargin=3,
frametitlerule=true,frametitlerulecolor=green,linewidth=0.9pt,
frametitlebackgroundcolor=yellow,
frametitlerulewidth=2pt}

\newtheoremstyle{break}% name
  {}%         Space above, empty = `usual value'
  {}%         Space below
  {\itshape}% Body font
  {}%         Indent amount (empty = no indent, \parindent = para indent)
  {\bfseries}% Thm head font
  {.}%        Punctuation after thm head
  {\newline}% Space after thm head: \newline = linebreak
  {}%         Thm head spec
\theoremstyle{break}
\newtheorem{theorem}{Theorem}
\newtheorem{lemma}[theorem]{Lemma}
\newtheorem{corollary}[theorem]{Corollary}
\newtheorem{prop}{Proposition}
\theoremstyle{definition}
\newtheorem{definition}{Definition}


\DeclarePairedDelimiter{\abs}{\lvert}{\rvert}
\DeclarePairedDelimiter{\norm}{\lVert}{\rVert}

\newcommand{\E}{\mathbb{E}}
\newcommand{\Var}{\mathrm{Var}}
\newcommand{\Cov}{\mathrm{Cov}}



\sisetup{output-decimal-marker = {,}}
\newcommand*{\ft}[1]{_\mathrm{#1}} 
\newcommand*{\dd}{\mathop{}\!\mathrm{d}}
\newcommand*{\tran}{^{\mkern-1.5mu\mathsf{T}}}%transpose of matrix
\newcommand{\trace}{\mathrm{trace}}

%%new
\newcommand{\tab}{\hspace{.2\textwidth}}
%\newcommand{\span}{\mathrm{Span}}
\renewcommand{\baselinestretch}{1.5}

%%%indenting
%\newlength\tindent
%\setlength{\tindent}{\parindent}
%%%\setlength{\parindent}{0pt}
%\renewcommand{\indent}{\hspace*{\tindent}}


%quotes
\newcommand{\citeauthorandyear}[2][]{
   \citeauthor{#2} (\citeyear[#1]{#2})
}
\newcommand{\argmin}{\mathrm{argmin}}
%
\title{DC programming}
\author{Frédéric Boileau}
\date{}
\addbibresource{dcproject.bib} 
\pagestyle{fancy}	
\begin{document}
\maketitle
\tableofcontents
\clearpage

\section{Introduction}

\subsection{TODO}
\begin{enumerate}
	\item Analysis of DCA, rate, stability, optimality, local vs global
	\item specific cases (Generalized Fermat-Weber)
	\item Using relative interior notions define other optimality
		criteria, "add structure for finite dimensional case
	\item Analysis of the sets DC and DCf, starting with list
		of elementary options under which they are closed 
		and after more subtle limiting arguments, e.g. smooth
		functions are DC
	\item Explorations: Hilbert spaces, proximal methods, accelerated 
		descent methods etc.
\end{enumerate}
\begin{comment}
	The field loosely defined as continuous optimization has had different
	stages in the last century that might appear disconnected to the
	uninitiated. In the first half properties of convex functions and sets
	are studied by Minkowski and Farkas, not obscuse mathematicians by any
	measure but optimization stays anchored to two main pillars, the
	combinatorial case which is studied in tandem with computer science
	which is at its inception and the linear case which is now a well
	established "technology" to use Boyd's terms. The rest of continuous
	optimization kept revolving around ancient ideas because of the
	constraints imposed by differentiability or smoothness, and even more
	because numerical simulations are still at their prehistorical state.
	While combinatorial optimization has some nice connections with graph
	theory which is also emerging and theoretical computer science, the
	optimization of continuous functions seems at the time like a mere
	mixture of numerical techniques, quite un-aesthetic often and doesn't
	seem to connect to other theoretical results.  The modern theory arises
	out of the works of Rockafellar and Fenchel amongst others who shifted
	the emphasis towards \emph{geometrical} concerns, explored duality that
	is the offspring in some regard's of Von Neumann's minimax theorem and
	constructed tools to approach analysis in non smooth contexts with
	subdifferentiability and in general established the grounds for
	continuous optimization to be a mathematical field in its own right.\\

It is of notable interest that Rockafellar's first book is called "Convex
Analysis" while his magnum opus published decades later adopts the title
"Variational Analysis". Now an increasing number of monographs get published
with tags such \emph{non-linear optimization} or \emph{variational problems}.
It testifies to the fact that non-linear optimization is starting to grow out
of convex analysis per-se.  Convex analysis initially provided the rigorous
foundation needed to approach problems that didn't present some of the
structures on which mathematical research had been focusing so far, linearity,
differentiability and integrability being the three major ones. It allowed to
do some nice mathematics even when both linearity and differentiability
vanished with an elegant geometrical framework. This approach can be summarized
by an important and quite simple result: every convex function can be
identified with its epigraph.\\
\end{comment}
In this paper we investigate a type of non-convex program which allows us to
leverage the tools from convex analysis in an elegant manner and which has many
direct concrete applications. We are concerned with the minimization of the
difference of convex functions and we use the following notation throughout:
\begin{equation} \lambda := \min_{x\in\mathbb E}\{ g(x)-h(x)\}\end{equation}
We usually assume that both $g$ and $h$ are convex, proper and lower 
continuous unless otherwise specified so.

\clearpage
\section{Preliminaries}
We define a couple of key concepts. 
First for exposition we use the following function notation throughout:
\begin{equation*}
	f: \mathbb R^n \longrightarrow \overline{\mathbb R} 
\end{equation*}
We also use Rockafellar's notation \autocite{rockafellar2009variational} for 
sequences of real variables : $x^{\nu}$ where $\nu$ is always understood 
to be a natural number used as an index and not an exponent.
\begin{definition}
A function's \emph{domain} is the subset of the initial set 
of the function where it doesn't attain infinity. Rockafellar
and other's sometimes refer to this set as the \emph{effective} domain
to emphasize the difference with the naive set theory definition. However
we will always refer to the domain in the first sense and will dispense from
using the adjective "effective". Formally:
\begin{equation*}
	\ddom f := \{x\in\mathbb R^n \ |\ f(x)< \infty\}
\end{equation*}
\end{definition}
%
\begin{definition}
	A convex function is said to be \emph{proper} if it has non empty domain
	and $f(x)>-\infty$ for all $x\in\mathbb R^n$. The set of all such
	functions is denoted $\Gamma$.
\end{definition}

\noindent In optimisation smoothness assumptions often break-down. We thus need
a different machinery than the one obtained from classical calculus. We
introduce some of those notions now. 

\begin{definition}
We define present the definition of the lower limit of a function $f$ at some
$\bar x\in\mathbb R^n$ as it is written by Rockafellar and Wets
\autocite{rockafellar2009variational}
\begin{align*}
	\liminf_{x\rightarrow \bar x} f(x) :&= \lim_{\delta \searrow 0}
	\left[\inf_{x\in B(\bar x,\delta)}f(x)\right]\\[2ex]
		&=\sup_{\delta >0}\left[\inf_{x\in B(\bar x,\delta)}f(x)\right]
		\quad = \sup_{V\in \mathcal{N}(\bar x)}\left[\inf_{x\in
		V}f(x)\right]
\end{align*}
\end{definition}
\begin{prop}
	\autocite{rockafellar2009variational}\\
The above definition of lower limit of a function is equivalent to
\begin{equation*}
	\liminf_{x\rightarrow\bar x}f(x) = \min\{\alpha\in\overline{\mathbb R}\ 
		| \ \exists x^{\nu} \rightarrow \bar x \ \text{with}
	\ f(x^{\nu})\rightarrow \alpha\} 
\end{equation*} We use min on the rhs as the definition presumes that the value
is actually attained.
\end{prop}
%
\begin{definition}
The concept of lower-limit allows us to talk about semicontinuity. 
A function $f$ is said to be \emph{lower semicontinuous} (lsc) at $\bar x$  if
\begin{equation*}
	\liminf_{x\rightarrow\bar x}f(x) \geq f(\bar x) 
	\qquad \text{or equivalently} \qquad 
	\liminf_{x\rightarrow\bar x}f(x)=f(\bar x)
\end{equation*}
Where the equivalence comes from the fact that the reverse inequality always
holds by definition of $\liminf$
\end{definition}
%
\begin{definition}
Recalling the definition of proper convex functions; when $f\in\Gamma$ and is
lsc we say that $f\in \Gamma_0$ which is set the of all such functions.
\end{definition} Those sets being of major importance we restate, more
formally:
\begin{equation*}
	\Gamma:= \{f:\mathbb R^n \longrightarrow \mathbb \overline{\mathbb R} \
	| \ f \ \text{proper and convex}\}
	\qquad \Gamma_0 := 
	\{f:\mathbb R^n \longrightarrow \mathbb \overline{\mathbb R} \
	| \ f \ \text{proper , lsc and convex}\}
\end{equation*}
%
\begin{definition}
	Let $f: \mathbb R^n \longrightarrow \overline{\mathbb R}$ be convex and
	$x\in\mathbb R^n$. Then $g\in\mathbb R^n$ is called a subgradient of
	$f$ at $\bar x$ if 
	\begin{equation*}
		f(x) \geq f(\bar x) +\langle g,x-\bar x\rangle \quad 
		\forall x\in\mathbb R^n 
	\end{equation*}
	Moreover the set of all such subgradients of $f$ at $\bar x$ is callled
	the \emph{subdifferential} of $f$ at $\bar x$ and we denote it as
	$\partial f(\bar x)$.
\end{definition}
%
The  \emph{$\epsilon$-subgradient} of a function is a generalization 
of the subgradient and  has been mostly developped by J.B. Hiriart-Urruty
in the second of his two volumes on convex analysis
\autocite[92]{hiriart1993convex} 
\begin{definition} 
	Given $x\in\ddom f$ the vector $s\in \mathbb R$ is called
	an $\epsilon-subgradient$ of $f$ at $x$, written $s\in
	\partial_\epsilon f(x)$ when the following holds
	\begin{equation*}
		f(y) \geq f(x) + \langle s,y-x\rangle -\epsilon 
		\quad \forall \ y \in \mathbb R^n \quad
	\end{equation*}
The set of all $\epsilon$-subgradients of a function at some point
$x\in\mathbb R^n$ is called the $\epsilon$-subdifferential of $f$ at $x$.
Moreover it is clear from the definition that
$\partial f(x) = \bigcap_{\epsilon>0}\partial_{\epsilon}f(x)$
\end{definition}
%
\begin{definition}
The Fenchel conjugate of a function $f$, in the convex analysis setting often
just referred to as the conjugate of a function, is an extended real valued
functione defined as:
\begin{equation*}
	f^*(y) := \sup_{x\in\mathbb R^n}\{\langle x,y\rangle - f(x)\}
\end{equation*}
The mapping $f\mapsto f^*$ is called the Legendre-Fenchel transform.
It is sometimes just refered to as the Legendre transform in physics which 
doesn't do justice to Fenchel's important work in the non-differentiable
and variational case.
\end{definition}
\noindent By definition we always have that
\begin{equation*} f(x)+f^*(y)\geq\langle x,y\rangle \quad \forall\ x,y\in 
\mathbb R^n\end{equation*}
This relationship is known as the \emph{Fenchel-Young Inequality}


\begin{lemma}
	Let $f\in\Gamma$ and $x^o \in\ddom f$ then we have that
	$y \in\partial_\epsilon f(x^o)$ if and only if 
	$f(x^o)+f^*(y) \leq \langle x^o,y\rangle + \epsilon$ 
	\label{FYepsilon}
\end{lemma}
\begin{proof}
	\begin{align*}
		y\in\partial_\epsilon f(x^o) &\iff 
		f(x) \geq f(x^o) + \inp{y}{x-x^o} - \epsilon
		\qquad x\in\mathbb R^n \\
		&\iff \epsilon + \inp{y}{x^o} \geq f(x^o) 
		+\underbrace{\sup_x \{\inp{x}{y} - f(x)\}}_{f^*(y)}
		&\iff \epsilon + \inp{y}{x^o} \geq f(x^o) + f^*(y)
\end{align*}
\end{proof}
\begin{lemma}\label{fenchelDualLemma}
	Let $f\in \Gamma$, then $y\in \partial f(x) \iff 
	f(x) + f^*(y) = \langle x,y \rangle $
\end{lemma}
\begin{proof}
    Since we already have that $f(x)+f^*(y)\geq\langle x,y\rangle \quad \forall\ x,y\in\mathbb R^n$ (Fenchel-Young)
    we only need the reverser inequality.\\
Let $y \in \partial f(x)$ 
\begin{align*}
    y \in \partial f(x) &\iff y \in \bigcap_{\epsilon>0}\partial_{\epsilon}f(x)\\
    \text{(By the preceding lemma)} \qquad 
    &\iff f(x) + f^*(y) \leq \inp{x}{y} + \epsilon \qquad \forall \ \epsilon > 0 \\
    &\iff f(x) + f^*(y) \leq \inp{x}{y} 
\end{align*}
\end{proof}
\begin{lemma}\label{lemma:neighToGlobal}
  Let $f$ be a convex function. Then provided that the following
  holds:
  \begin{equation}
    f(x) \geq f(\bar x) + \inp{g}{x-\bar x} 
  \end{equation}
  for all $x$ in some some neighborhood of $\bar x$ then it actually
  hold for all $x \in \mathbb R^n$ and $g$ is thus a subgradient of
  f at $\bar x$
  \begin{proof}
    Let $N(\bar x)$ designate the neighborhood of $\bar x$ for which
    the inequality holds. Let $\lambda >0$ be small enough 
    so that $(\lambda x+ (1-\lambda)\bar x) \in N(\bar x)$
    By convexity we have :
    \begin{align}
%
      \lambda f(x) + (1-\lambda)f(\bar x) &\geq
      f(\lambda x + (1-\lambda)\bar x) \qquad \forall \ x \in \mathbb R^n  \\
      &\geq f(\bar x) + \inp{g}{[\lambda x + (1-\lambda)\bar x)] - \bar x} 
      \qquad \forall \ x \in \mathbb R^n 
%
    \end{align}
  Letting $\lambda$ go to zero gives us the desired inequality.
  \end{proof}
\end{lemma}
\clearpage

\begin{definition}
        A convex function is said to be  strongly convex with modulus
	$\mu>0$ if 
        \begin{equation}
          f(x) - \frac{\mu}{2}\norm{x}^2
        \end{equation}
        defines a convex function.\\
\end{definition}

\begin{lemma}
    If a function $f$ is strongly convex with modulus $\mu>0$  then 
  $y\in \partial f(\bar x)$ if and only if:
  \begin{equation}
    f(\bar x) \leq f(x) + \inp{y}{x-\bar x} - \frac{\mu}{2}\norm{x-\bar x}^2
  \end{equation}
  \begin{proof} 
      Let $y \in \partial f(\bar x)$
      Let $f$ be strongly convex with modulus $\mu>0$, i.e. $g(x) = f(x) -
      \frac{\mu}{2}\norm{x}^2$ defines a convex function. Then by definition of convexity 
      we have :
      \begin{align}
	  f(\lambda x + (1-\lambda)y) &\leq \lambda f(x) + (1-\lambda)f(y) 
	  + \frac{\sigma}{2}(\norm{\lambda x + (1-\lambda y)}^2 - \norm{x}^2 -
	  (1-\lambda)\norm{y}^2)\\
%
	  &\leq \lambda f(x) + (1-\lambda)f(y) -
	  \frac{\sigma}{2}\lambda(1-\lambda)\norm{x-y}^2 
      \end{align}
\end{proof}
\end{lemma}









\clearpage
\section{Duality and Optimality}
\subsection{Toland-Singer duality}
Let us recall the definition of a DC program and its associated notation.
\begin{equation*}
	\lambda := \min_{x\in\mathbb E}\{ g(x)-h(x)\}
	\qquad g,h\in\Gamma_0
\end{equation*}
Let $f:=g-h$ be the objective function of the above. We write 
$\mathcal{DC}$ for the set of all such objective functions and 
$\mathcal{DC}_f$ when they are finite. \underline{TODO: DCf closed under}\\

With the DC program defined the first important result to establish is duality.
It turns out there is a very natural dual to P that is due to Toland and Singer.
Letting our primal problem as defined previously:
\begin{alignat*}{4}
	&P \qquad \lambda &&:=\inf_{x\in\mathbb E} \{g(x) - h(x)\} \qquad
	&&\mathcal{P} := \argmin_{x\in\mathbb E}\{g(x)-h(x)\}
	\qquad&& g,h \in \Gamma_0
	\intertext{We define the dual as follows:}
	&D \qquad \lambda^* &&:=\inf_{y\in\mathbb E} \{h^*(x) - g^*(x)\} \qquad
	&&\mathcal{D} := \argmin_{y\in\mathbb E}\{h^*(y)-g^*(y)\}
	\qquad&& g,h \in \Gamma_0
\end{alignat*}
Letting once again $f:=g-h$ we write $f^\dagger$ for $h^*-g^*$ and refer
to that function as the DC-dual of $f$. Note that in order to have
a solution to the program (minimizer where the objective value is not $\infty$)
we clearly need $x \in \ddom g$, as even if we have the pathological case
that both $x\not\in\ddom g$ and $x\not\in\ddom h$ then $f(x)= \infty
-\infty = + \infty $ under the inf-addition convention. \emph{Hence we assume
$x\in\ddom g$ unless otherwise specified.} We know present the first and
probably most important result on DC programming due to Toland-Singer which
establishes how a DC program is related to its DC-dual. The proof we present is
from Hoheisel's notes on convex analysis.  \autocite{notes}.
\clearpage
\begin{theorem}
	a)Let $\lambda$ and its dual be defined as above, we have that
	$\lambda=\lambda^*$\\
	b)Moreover we have that given an $\bar x\in \mathcal P$ every $y\in
	\partial h (\bar x)$ is in $\mathcal D$.
\end{theorem}
\begin{proof}
	a) Since $h\in \Gamma_0$ we directly have that h is equal to its
	biconjugate hence
\begin{align*}
	\inf_x\{g(x) - h(x)\}
		&= \inf_x\{g(x) - h^{**}(x)\}\\
		&= \inf_x\{g(x) - (h^*)^*(x)\}\\
		&= \inf_x\big\{g(x) - \sup_y\{ \langle x,y \rangle - h^*(y)\}\big\}
\end{align*}
Now let us note that 
\begin{equation}
	\inf_x\big\{g(x) - \{ \langle x,y \rangle - h^*(y)\}\big\} < +\infty\
	\iff \ y \in \ddom h^*
\end{equation}

\begin{align*}
		\inf_x\{g(x) - h(x)\}	&= \inf_x\big\{g(x) + \inf_y\{
			h^*(y) - \langle x,y \rangle \} \big\}\\
		&= \inf_x \inf_y \big\{ h^*(y) - \langle x,y \rangle +
		g(x)\big\}\\
		&= \inf_y \big\{ h^*(y) - \sup_x\{\langle x,y\rangle
			-g(x)\}\big\}\\
		&= \inf_y \{h^*(y) - g^*(y)\}
\end{align*}

\noindent Hence the finiteness of $\lambda$ boils down to 
\begin{equation}
	\ddom  g \subset \ddom h \qquad\text{and}\qquad \ddom h^* \subset \ddom
	g^* \label{finiteassumption}
\end{equation}
We thus assume \ref{finiteassumption} holds throughout the paper. Let us continue
the computations, now anchored in the finite setting :
\noindent b) By assumption $\bar x \in \mathcal P$ hence by definition:
\begin{equation*}
	g(\bar x)-h(\bar x) \leq g(x) - h(x)
\end{equation*}
Under the assumption that $x\in\ddom g$ and the inclusions of
\ref{finiteassumption}
all terms are finite and we can re-arrange:
\begin{equation*}
	g(x)-g(\bar x) \geq h(x)-h(\bar x)
\end{equation*}
Note that this would still be valid in the case where either just 
the LHS or both sides were infinite valued under the inf-addition convention;
which is actually the main motivation behind said convention.
Now using the assumption that some $\bar y$ is in the range of $\partial h(\bar
x)$ and the definition of the subdifferential
\begin{align*}
	g(x)-g(\bar x) \geq \langle\bar y,x-\bar x\rangle  \quad
	\Rightarrow \qquad 
	\bar y \in\partial g(\bar x) 
	\quad (\text{and} \ y\in  \partial  h(\bar x)\ \text{by assumption})
\end{align*}
Recall that whenever $\bar z\in\partial f(z)$ for some $f\in\Gamma$, lemma
\ref{fenchelDualLemma} gives us the following equality $ f(z)+f^\dagger(\bar z) =
\langle z,\bar z\rangle $. This leads to the following observation: the point
$\bar y$ being in the range of the subdifferentials of both $g$ and $h$ "binds"
the value of the DC program and its DC-Dual.
\begin{equation*}
	g(\bar x) + g^*(y) = h(\bar x) + h^*(\bar y) = 
	\langle \bar x,\bar y \rangle \qquad 
\end{equation*}
Where the inner product is less relevant here than the fact that the value agree.
Hence all in all we get :
\begin{equation*}
	g(\bar x) - h(\bar x) = h^*(\bar y) - g^*(\bar y)
\end{equation*}
\end{proof}
Note that the assumption on $g\in \Gamma_0$ is unnecessarily strong and the two
previous proofs work by replacing it  by any proper extended real valued
function. However if we want a symmetric result, that is  $\bar y \in
\mathcal D$ and $\bar x \in \partial g^*(\bar y)$ imply $\bar x \in \mathcal P$,
then we do need $g$ to be l.s.c. and the proof is essentially the same as the
one above by symmetry.
\clearpage

\subsection{Optimality Conditions}
\underline{TODO}:Global optimality criteria with relative interiors 
\autocite{tao2005dc}\\

As mentionned previously global optimality is not within our
reach in practical applications at this point and we usually have to resort to
use local optimality criteria to derive algorithms eventhough they might 
in fact converge to a global solution. I now present different theorems
and conditions for global optimality. To make sure both $f$ and its DC-dual
are finite valued we assume the following:
\begin{equation}
	\ddom g \subset \ddom h \qquad\text{and}\qquad
	\ddom   h^* \subset \ddom g^*
\end{equation}

\begin{theorem}[Characterization of global minima]
A point $\bar x\in \mathbb R^n$ is a solution to the primal
problem, i.e. $\bar x\in \mathcal P$, if and only if 
$\partial_\epsilon h(\bar x) \subset \partial_\epsilon g(\bar x)\quad \forall
\epsilon >0$
\end{theorem}
\begin{proof}
Assume $x\in\ddom g$ and $y\in\ddom h^*$ for finiteness.
Let $\bar x \in \mathcal P$. Since by DC-duality
the primal and dual optimal values agree we have that  $f(\bar x) \leq
f^\dagger(y)$, hence : 
\begin{alignat}{2}
	&g(\bar x) -h(\bar x) &&\leq h^*(y)-g^*(y)\\
	\iff \qquad &g(\bar x) + g^*(y) &&\leq h(\bar x) + h^*(y) 
		     \label{globalCondition}
\end{alignat}
%
Now let $\bar x\in \mathcal P$, from Fenchel-Young inequality we have:
\begin{equation}
	\inp{\bar x}{y} \leq g(\bar x)+g^*(x)\leq h(\bar x)+h^*(x) \qquad
	\forall x\in\ddom g
\end{equation}
Then if $x\in\partial_\epsilon h(\bar x)$ we have from "generalized 
Fenchel-Young" inequality (Lemma \ref{FYepsilon}) that:
\begin{equation*}
	h(\bar x)+ h^*(x) \leq \epsilon + \inp{x}{\bar x}
\end{equation*}
Then clearly if we want \eqref{globalCondition} to potentially hold we need to
have $g(\bar x)+ g^*(x) \leq \epsilon + \inp{x}{\bar x}$
as well. Hence $x\in \partial_\epsilon g(\bar x)$ whenever $x\in
\partial_\epsilon h(\bar x)$ which shows the inclusion is necessary. For 
sufficiency we argue by contraposition. Assume \eqref{globalCondition} does not
hold. Then by Fenchel-Young there exists
some $\epsilon>0$ such that
\begin{equation*}
	\inp{x}{\bar x} \leq h(\bar x)+h^*(x)< \inp{x}{\bar x} + \epsilon 
	\leq g(\bar x) + g^*(x) \qquad \forall x\in\ddom g
\end{equation*}
But then once again by lemma \ref{FYepsilon} we have that
$x\in\partial_\epsilon h(x)$ and $x\not\in\partial_\epsilon g(x)$
\end{proof}
\clearpage

An important remark is in order, notationnally the
usage of $\epsilon$ has a heavy connotation as we expect to let it go to zero
or to use it to get the strongest estimate possible. However here letting it to
zero yields the much weaker condition $\partial h(x) \subset \partial g(x)$
The later actually yields a condition which while not sufficient is clearly
\emph{necessary} for global optimality.The strength of the condition imposed
with the $\epsilon-$subdifferential for all $\epsilon>0$ lies in the fact that we
can also make $\epsilon$ as large as we'd like. This necessary and sufficient
condition is in fact too strong and impractical as are gobal optimality
conditions in general.  Now I recall the property from the proof DC-duality
that
\begin{equation}
	\forall  \bar x \in \mathcal P \quad \partial h(\bar x)  \subset
	\mathcal D  \subset  \ddom h^* \qquad \forall \bar y \in \mathcal D
	\quad \partial g^*(\bar y) \subset  \mathcal P \subset \ddom g	
\end{equation}
This property is commonly known as \emph{transportation of global minimizers}
as subdifferentials map  minimizers between primal and dual spaces. This
is a very good guide for our intuition as I will prove the next couple results
about \emph{local optimality} which are at the core of the DCA. First we
introduce the following
notation due to Tao and Souad \autocite[280]{tao1988duality}:
\begin{equation}
	\mathcal P_l := \{\bar x\in \mathbb R^n\ |\
	\partial h(\bar x)\subset\partial g(\bar x)\} \qquad
	\mathcal D_l := \{\bar y\in \mathbb R^n\ |\ 
	\partial g^*(\bar y) \subset \partial h^*(\bar y)\}
\end{equation}

\begin{theorem}[Necessary condition for local optimality]
If $\bar x$ is a local minimizer of $f:=g-h$ then $\bar x \in \mathcal P_l$
\begin{proof}
If $\bar x$ is a local minimizer of $f$ then by definition there
exists a neighborhood of $\bar x$, say $N(\bar x)$ such
that 
%
\begin{equation}
	g(x)- h(x) \geq g(\bar x) - h(\bar x) \quad 
	\forall x\in N(\bar x)  
\end{equation}
\noindent Since by assumption $g$ is proper we can
take the intersection of the neighborhood with $\ddom g$ without yielding an
empty set. Taking this intersection forces $g$ and $h$ to be finite valued
\footnote{Recall that we always assume $ \ddom g \subset \ddom h$}
so that we can rearrange the former inequatlity:
\begin{equation}
	g(x)- g(\bar x) \geq h( x) - h(\bar x) \quad 
	\forall x\in N(\bar x) \cap \ddom g  \qquad 
	\label{localNec-1}
\end{equation}
\noindent Now for any $\bar y \in\partial h(\bar x)$ we have that 
\begin{equation}
	h(x)-h(\bar x) \geq \langle x-\bar x, \bar y\rangle 
	\qquad x\in\mathbb R^n
	\label{localNec-2}
\end{equation}
%
Combining \eqref{localNec-1} and \eqref{localNec-2} we get
that given some   $\bar y \in \partial h(\bar x)$:
%
\begin{equation}
	g(x)-g(\bar x)\geq \langle x-\bar x,\bar y\rangle \quad \forall \ 
	\bar x \in N(\bar x)\cap \ddom g 
	\label{localNec-3}
\end{equation}
Note that the set $N(\bar x) \cap \ddom g$ might not be open for an abritrary
neighborhood however we can clearly make $N(\bar x)$ small enough so that its
intersection with $\ddom g$ is open. Finally proving the inequality 
\eqref{localNec-3} on some neighborhood for a convex function actually
implies it holds for all of $\mathbb R^n$ (Lemma) hence :
\begin{gather*}
	g(x)-g(\bar x)\geq \langle x-\bar x,\bar y\rangle \quad \forall \ 
	\bar x \in \mathbb R^n\\
	\therefore \  \bar y \in\partial g(\bar x)
\end{gather*}
All in all we have just proven that local minimality implies the desired
inclusion, i.e. that  $\bar x \in \mathcal P_l$. 
\end{proof}
\end{theorem}
\clearpage

\begin{definition}[Critical Point]
	A point $\bar x$ is said to be a \emph{critical point of $g-h$} if
	\[\partial g(\bar x) \, \cap \, \partial h(\bar x) \neq \emptyset \]
\end{definition}
We now proceed to show a sufficient condition for local optimality.
We first present the main result and then a more elegant corollary
to test local optimality. Both results are drawn from  Tao and An's paper
\autocite{tao1997convex}.
First we need a critical point and then explore the behaviour of 
the subdifferentials of $g$ and $h$ around it. In the following proof we
restrict the neighborhood to the domain of the subdifferential of $h$ as this
mapping allows us to exploit the dual structure of DC programs.

%
\begin{theorem}[Sufficient local optimality condition]
%
Let $\bar x$ be a critical point of $f:=g-h$ , 
$\, \bar y \in \partial g(\bar x) \cap \partial h(\bar x)$ and 
$\widetilde{N}(\bar x):= N(\bar x) \cap \ddom
\partial h$. Then, given the definitions above, we have that if 
\begin{equation}
	\forall x \in \widetilde{N}(\bar x) \quad \exists \
	\hat y \in \partial h(x)  \quad s.t.\quad
	f^\dagger(\hat y)\geq f^\dagger(\bar y)
\end{equation}
Then  $\bar x$ is a local minimizer on $\widetilde N(\bar x)$
\begin{proof}
Once again taking the intersection of both subdifferentials
allows us to bind the primal and dual problems. 
\begin{equation*}
	\bar y \in \partial g(\bar x) \cap h(\bar x) \ \Rightarrow 
	g(\bar x)+g^*(\bar y)=\langle \bar x,\bar y\rangle = 
	h(\bar x) + h^*(\bar y)
\end{equation*}
Hence  we get the equality of the primal and dual objectives
\begin{equation}
	g(\bar x) - h(\bar x) = h^*(\bar y) - g^*(\bar y)
	\label{localsuf1}
\end{equation}
Moreover we have that $\forall x \in \widetilde{N}(\bar x)$ there is 
$\hat y\in \partial h(x)$ which majorizes the dual at $\bar y$, i.e.
\begin{equation}
	h^*(\hat y) - g^*(\hat y) \geq h^*(\bar y) - g^*(\bar y) \qquad
	\text{In other words} \quad f^\dagger (\hat y) \geq f^\dagger(\bar y)
	\label{localsuf2}
\end{equation}
To complete the proof we need one last inequality which is direct
from the Lemma \ref{fenchelDualLemma} (as $\hat y \in \partial h(x)$)
and the definition of the Fenchel dual of a function:
\begin{equation}
	h(x)+h^*(\hat y) = \langle x,\hat y\rangle \leq g(x)+g^*(\hat y)
	\ \Rightarrow \ g(x)-h(x)\geq h^*(\hat y) -g^*(\hat y)
	\label{localsuf3}
\end{equation}
Combining \eqref{localsuf1}, \eqref{localsuf2} and \eqref{localsuf3}
we get the desired inequality:
\begin{equation*}
	g(x) - h(x) \geq g(\bar x)-h(\bar x) \qquad \forall x \in 
	\widetilde N(\bar x) 
\end{equation*}
\end{proof}
\end{theorem}
\clearpage
We can now leverage this rather technical result to get the following more
elegant corollary which follows quite nicely. The idea however is quite in the
same vein. For a critical point $\bar x$ we need a neighborhood around it that
is contained in the subdifferential of $g$ at $\bar x$ and such that every point
of the neighborhood is a fixed point under the subdifferential of $h$.
\begin{corollary}[sufficient local optimality condition]
	Let $\bar x$ be a critical point of ($\ \mathcal{DC}\ni)\ f:=g-h$.  If
	we can find a neighborhood $\widetilde{N}(\bar x):=N(\bar
	x) \cap \ddom\partial h(x)$ such that
\begin{equation}
	\forall x \in \widetilde{N}(\bar x) \qquad \partial g(\bar x) 
	 \cap \partial h(x) \neq \emptyset
\end{equation}
	then $\bar x$ is a local minimizer of f
\begin{proof}
	Let $x\in \widetilde{N}(\bar x)$ and $y\in  \partial h(x) \ \cap
	\partial g(\bar x) $\\[1ex] First we consider $\bar y \in \partial
	h(\bar x) \cap \partial g(\bar x)$ and as before it binds dual and
	primal
	%
\begin{align*}
	g(\bar x)+g^*(\bar y) &= \langle\bar x,\bar y\rangle 
	= h(\bar x)+h^*(\bar y) \ \Rightarrow f(\bar x)=f^\dagger(\bar y)
	%
	\intertext{Now we have two inclusions which give us
	two inequalities, first $y \in \partial h(x)$ yields}
	%
	h(x)+h^*(y)&=\langle x,y\rangle \leq g(x)+g^*(y) \
			\Rightarrow \ f(x) \geq f^\dagger(y)
	%		
	\intertext{And $y\in \partial g(\bar x)$ hence}
	%
	g(\bar x)+g^*(y)&=\langle \bar x,y\rangle 
		\leq h(\bar x)+h^*(y) \ \Rightarrow \ f^\dagger(y) \geq f(\bar x)
\end{align*}
Hence combining the three relationships above  we get 
\begin{equation*}
	f(x) \geq f(\bar x) \quad  \forall x\in \widetilde{N}(\bar x) 
	\qquad(:=N(\bar x) \cap \ddom\partial h(x))
\end{equation*}
\end{proof}	
\end{corollary}
For both the theorem and its corollary the process of restricting the neighborhood
to the domain of the subdifferential of $h$ allows us to get satisfying 
inequalities in the dual space which can be linked elegantly to the primal. 
The corollary requires stronger assumptions as the neighborhood is not only
restrained to $\ddom \partial h(x)$ but also to the subdifferential of $g$ at
the critical point.
\clearpage

\section{DCA}
\newcommand{\xs}{x^\nu}
\newcommand{\xss}{x^{\nu+1}}
\newcommand{\ys}{y^{\nu}}
\newcommand{\yss}{y^{\nu+1}}
\subsection{Intuition and Derivation of simplified DCA}
We now have the technical results necessary to discuss the
DC algorithm for general D.C. programs. Its aim is to 
generate sequences $x^\nu$ and $y^\nu$ which converge to a local
minimum of the functions $f:=g-h$ and $f^\dagger:= h^*-g^*$
respectively where $g,h\in\Gamma_0$\\ We define two sub programs:
\begin{align} 
 \widetilde S(x) \qquad &\inf\{h^*(y)-g^*(y) \ : \ y\in\partial h(x)\}\\ 
 \widetilde T(y) \qquad &\inf\{g(x)-h(x) \ : \ x\in\partial g^*(y)\}
 \label{fulldca}
\end{align}
Let $\widetilde {\mathcal S}(x)$ and $\widetilde{\mathcal T}(y)$ denote the
solution sets of $\widetilde S(x)$ and $\widetilde T(y)$ respectively. Then given
a point $x^o\in\ddom g$
the sequences are constructed as follows:
\begin{equation}
  \ys\in\widetilde{\mathcal S}(\xs); \quad \xss\in\widetilde{\mathcal T}(\ys)
\end{equation}
The main idea relies on a decomposition approach to the problem;
we use DC duality to solve the initial program and its dual
on subsets of the initial space using the subdifferential
operators to move between primal and dual. Eventhough the 
the subprograms are simpler than the initial one they are 
still non-convex. Their main function is to give 
insights as to how to construct sequences with the desired
behaviour. \\

Consider the program $T(\ys)$ where $\ys$ is the solution
to some $\widetilde S(x^o)$. We have that $\ys\in\partial h(\xs)$ hence:
\begin{equation}
    h(\xs)\leq h(\xss) + \inp{y}{x-\xss} 
\end{equation}
This suggests a simplfication of the program $\widetilde T(\ys)$ by replacing
$h(\xs)$ by its affine minorization through $\ys$ whereby we get the
following program, which is convex: 
\begin{equation}
    \inf\{g(x) - [h(\xs)+\inp{\ys}{x-\xs}] \}
    \label{simDCApp}
\end{equation}
Dually we can also replace $g^*$ in the objective function
of $\widetilde S(\xss)$ by its affine minorant since we have that
$\xss \in \partial g^*(\ys)$:
\begin{equation}
    \inf\{h^*(y)-[g^*(\ys) + \inp{\xss}{y-\ys}]\}
    \label{simDCAdd}
\end{equation}
Note that we only needed the inclusion $\xss\in\partial g^*(\ys)$
and $\ys\in\partial h(\xs)$ to find a simplified versions of
the two initial programs.\\
\begin{comment}
Note that the simplified forms \eqref{simDCApp} and \eqref{simDCAdd}
are actually equivalent to the programs from \eqref{fulldca} if the functions 
$g^*$ and $h$ are actually essentially differentiable \autocite{tao2005dc}.
\end{comment}
\newpage
%
The programs \eqref{simDCApp} and \eqref{simDCAdd} can actually be simplified
further as the objective functions contain constants.
%
\begin{equation}
   \argmin\{h^*(y)-[g^*(\ys) + \inp{\xss}{y-\ys}]\}=
   \argmin\{h^*(y) -\inp{\xss}{y}\}
\end{equation}
Now note that 
\begin{equation}
  \yss \in \argmin\{h^*(y) -\inp{\xss}{y}\} \iff \yss \in \partial h(\xss)
\end{equation}
Similarly
\begin{equation}
    \argmin\{g(x) - [h(\xs)+\inp{\ys}{x-\xs}] \}=
   \argmin\{g(x) -\inp{x}{\ys}\}
\end{equation}
\begin{equation}
  \xss \in   \argmin\{g(x) -\inp{x}{\ys}\} \iff \xss \in \partial g^*\ys)
\end{equation}
Hence all in all the simplified DCA 

\clearpage

%


\appendix
\clearpage
\printbibliography
\clearpage
\section{authors}
Pham dih tao\\
El Bernoussi\\
JB Hiriart-Urruty\\
R. Ellaia\\
JF Tolland (1979 important paper)

\begin{comment}

\clearpage
It could be argued that convex optimization is, to once again use
Boyd's terminology, a technology, and while it is a very useful and
beautiful tool many real life problems are indeed non convex which
prove to be very resistant to traditional techniques. In those cases
there are different approaches :
\begin{itemize}
	\item Convexification where we approximate the problem
		by a convex one and control the erro
	\item Brute force AI type optimization where we find
		heuristics informed by "real-life" knowledge
		of the problem to find solutions which are
		useful and seem quite correct but with no
		optimality guarantee whatsoever
	\item Carefully extend the tools from convex analysis to some
		well defined cases of where non-convexity arises.
\end{itemize}
In this paper I discuss the third case in the specific context of the difference 
of two convex functions. I assume knowledge of the basics of convex analysis such
as Fenchel duality, lower-semicontinuityo, sub-differentials, epigraphs and relative
interiors. I hope however that the treatment will illuminate and motivate some
of those motivations.  I will start by defining some notation which will be
used throughout the text.  \\The standard DC  Program is the following:
\begin{equation}
	P \qquad \lambda :=	\inf\{g(x) - h(x) \, | \, x\in \mathbb{R}^n\}
	\qquad
	g,h \in \Gamma_0
\end{equation}
At first read this program might seem quite arbitrary and simple. Of all the
ways to start exploring the non-convex realm why choose this one. It turns out
that there are three very good reasons to explore such a program. First and
foremost it is actually a very natural generalization and step outside of
convexity for reasons I will explore later. Moreover there are many real-life
problems which can be very easily modelled by such programs and finally a
powerful algorithm has been developped for DC programs (DCA algorithm) which
although it doesn't guarantee global optimality \emph{does} often converge to
global solutions and does so very robustly and efficiently. Those three factors
suggest the relevance of investigating those programs.  
\end{comment}


\end{document}

