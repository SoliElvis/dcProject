\documentclass{article}

\usepackage{comment}
\usepackage[english]{isodate}
\usepackage{graphicx}
\usepackage[margin=1in]{geometry}
\usepackage{paracol}
\usepackage[utf8]{inputenc}
\usepackage[T1]{fontenc}
\usepackage[bookmarks=true]{hyperref}
\usepackage{bookmark}
\usepackage{pdfpages}
%\includepdf[pages={1}]{myfile.pdf}

%quotes
\usepackage{csquotes} 
\usepackage{endnotes}
\usepackage[toc,page]{appendix}


\usepackage[backend=biber,style=authoryear]{biblatex} 
\usepackage[toc,page]{appendix}
\usepackage{bookmark} 

\renewcommand*{\thefootnote}{\alph{footnote}}

\usepackage[belowskip=-15pt,aboveskip=0pt]{caption,subcaption}
\usepackage{cancel}
\usepackage{listings}
\usepackage{fancyhdr}
\usepackage{lastpage}
\pagestyle{fancy}
\fancyfoot{}
\fancyfoot[R]{\thepage\ of \pageref{LastPage}}

\usepackage{mathtools,xparse}
\usepackage{amsmath}
\usepackage{amssymb}
\usepackage{siunitx}
\usepackage{amsthm}
\usepackage{amsbsy}

\usepackage{bm}
\newcommand{\ddom}{\mathrm{dom} \, }
\DeclarePairedDelimiterX{\inp}[2]{\langle}{\rangle}{#1, #2}

\usepackage{mdframed}
\mdfdefinestyle{exampledefault}{%
rightline=true,leftline=true, innerleftmargin=3,innerrightmargin=3,
frametitlerule=true,frametitlerulecolor=green,linewidth=0.9pt,
frametitlebackgroundcolor=yellow,
frametitlerulewidth=2pt}

\newtheoremstyle{break}% name
  {}%         Space above, empty = `usual value'
  {}%         Space below
  {\itshape}% Body font
  {}%         Indent amount (empty = no indent, \parindent = para indent)
  {\bfseries}% Thm head font
  {.}%        Punctuation after thm head
  {\newline}% Space after thm head: \newline = linebreak
  {}%         Thm head spec
\theoremstyle{break}
\newtheorem{theorem}{Theorem}
\newtheorem{lemma}[theorem]{Lemma}
\newtheorem{corollary}[theorem]{Corollary}
\newtheorem{prop}{Proposition}
\theoremstyle{definition}
\newtheorem{definition}{Definition}


\DeclarePairedDelimiter{\abs}{\lvert}{\rvert}
\DeclarePairedDelimiter{\norm}{\lVert}{\rVert}

\newcommand{\E}{\mathbb{E}}
\newcommand{\Var}{\mathrm{Var}}
\newcommand{\Cov}{\mathrm{Cov}}



\sisetup{output-decimal-marker = {,}}
\newcommand*{\ft}[1]{_\mathrm{#1}} 
\newcommand*{\dd}{\mathop{}\!\mathrm{d}}
\newcommand*{\tran}{^{\mkern-1.5mu\mathsf{T}}}%transpose of matrix
\newcommand{\trace}{\mathrm{trace}}

%%new
\newcommand{\tab}{\hspace{.2\textwidth}}
%\newcommand{\span}{\mathrm{Span}}
\renewcommand{\baselinestretch}{1.5}

%%%indenting
%\newlength\tindent
%\setlength{\tindent}{\parindent}
%%%\setlength{\parindent}{0pt}
%\renewcommand{\indent}{\hspace*{\tindent}}


\begin{document}
\section*{Fairness and efficiency in ressource allocation}
The setting is an optimization problem where we have some
ressources, denoted $i\in R$ and some agents, denoted $j\in J$.
The set of possible allocations $X$ lives in $\mathbb R^{m\times n}$.
Each agent $j$ derives some utility from acquiring some ressource, 
independently of what the others receive.
Say \begin{equation} U_j: \mathbb R^{m} \longrightarrow \mathbb R
\qquad (x_{j1},\ldots,x_{jm})\mapsto u_j\end{equation}
Most classical treatments of ressource allocation in economics focus on
efficiency, that is maximizing utility with little regard to "fairness"
considerations however in some fields the latter is critical.  For example
consider bandwith allocation to different links in a network where each
customer's service depend on different links. Maximizing overall bandwith
might result in some user being completely cutoff which is clearly
unacceptable. Different metrics of fairness have been proposed such as the
Gini and Atkinson indices and there is an associated set of
of properties and a well established axiomatic treatment of notions
of fairness in the litterature.
Moreover there is a tradeoff to be made between fairness and efficiency.
The overall objective function that can model both and prescribe
how one should dominate over the other is called a social welfare functionn.

I'd like to investigate the properties of the problem of optimizing a social
welfare function given a set of constraints and utility functions. More
specifically how the type of utility function and the similarity between them
for different individuals affects the structure of the problem under
classical considerations : convexity and smoothness. Moreover how 
do different choices of social welfare functions affect efficiency and 
fairness.

This is a setting extracted from a case a friend of mine was confronted with 
in the context of his financial consulting work.
When shareholders give some intentions to sell some of their stocks and
there is a maximal number of shares to sell, how does one distribute selling
options across the population in the fairest way. Increased complexity
is injected by the fact that there are different types of shares (e.g.
premium, normal) and agents hold those in different proportions.

Finally in computer heterogeneous  ressource allocation such as cloud computing,
the constraint and utility functions can be dynamic, parametrized by some 
$t\in \mathbb R_+$.
\end{document}
